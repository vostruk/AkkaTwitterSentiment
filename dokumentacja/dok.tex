\documentclass{article}

\usepackage{polski}
\usepackage[utf8]{inputenc}
\usepackage{graphicx}

\begin{document}


\title{Projekt SAG: analiza sentymentu w mediach społecznościowych}

\author{Maciej Lotz, Volodymyr Ostruk, Łukasz Wacławski}

\maketitle


\section{Analiza sentymentu}

\subsection{Źródło danych i model emocji}
Jako źródło danych treningowych posłuży Twitter, z którego dane będą pobierane przez Streaming API. Emocje w ściągniętych tweetach zostaną oznaczone na podstawie zawartych w nich znakach \emph{emoji}. Przyjęty został model 6 podstawowych emocji podany przez Paula Ekmana: złość, obrzydzenie, strach, szczęście, smutek, zaskoczenie.

Po wytrenowaniu modelu klasyfikacji użytkownik będzie mógł podać interesujący go hasztag oraz zakres dat (ograniczony do ostatnich 7 dni ze względu na możliwości API Twittera). W odpowiedzi program przedstawi użytkownikowi statystyki nt. emocji wyrażanych przez autorów tweetów w danym hasztagu w danym okresie.

\subsection{Algorytm klasyfikacji sentymentu}


\end{document}